\section{Experiments}

We report results on two frozen GPT-2 models (gpt2-medium and gpt2-large) using the repository protocols bundled with this paper.
All reported numbers are computed from the accompanying evidence JSON files (see Appendix~\ref{app:evidence_map}).

\subsection{Setup}
\paragraph{Models.}
We evaluate gpt2-medium as the primary model and gpt2-large as a second-model confirmation.
Inference uses batch size 1, eager attention, and the standard KV cache.

\paragraph{Tasks.}
We evaluate on:
(i) Wikitext-2 perplexity (400 examples),
(ii) SST-2 prompted classification (250 examples),
and (iii) a deterministic needle-style long-context probe (80 examples).
Exact prompting and subset sizes are specified in Appendix~\ref{app:eval_details}.

\paragraph{OOD protocol and corruption.}
We train CacheMedic++ on a fixed mixture of five corruption types and hold out \texttt{contiguous\_overwrite}.
Evaluation is performed on the held-out \texttt{contiguous\_overwrite} type at $\varepsilon\in\{0,0.08,0.16\}$ (LOTO; Sec.~\ref{sec:threat}).
For all runs, corruption applies only to the \texttt{old\_only} segment of the cache (all but the most recent 32 tokens), exposing long-range cache dependence.

\paragraph{Baselines and ablations.}
We compare against:
\textbf{no defense} and four inference-only heuristics (reset/clear, smoothing, masking/dropout, clipping/renorm).
We report \textbf{best heuristic} chosen by robustness AUC on the evaluation grid.
To test the stability framing, we include a \textbf{paired contraction ablation}: the same repair operator and training setup with $\lambda_{\mathrm{contr}}$ set to 0.

\subsection{Main results (gpt2-medium)}
Table~\ref{tab:main} summarizes clean score and robustness AUC.
CacheMedic++ improves both clean score and robustness AUC compared to no defense and the best heuristic baseline.
Figure~\ref{fig:robustness} shows the full robustness curve.

\begin{table}[t]
  \centering
  \caption{gpt2-medium results under OOD held-out \texttt{contiguous\_overwrite} (LOTO). Clean score is at $\varepsilon=0$; robustness AUC is trapezoidal area over $\varepsilon\in\{0,0.08,0.16\}$.}
  \label{tab:main}
  \vspace{2pt}
  \begin{tabular}{lrr}
\toprule
Method & Clean Score & Robustness AUC \\
\midrule
No defense & 0.2373 & 0.0386 \\
Best heuristic (smoothing) & 0.2373 & 0.0392 \\
CacheMedic++ & 0.2426 & 0.0392 \\
\bottomrule
\end{tabular}

\end{table}

\begin{table}[t]
  \centering
  \small
  \caption{Compact claim-to-evidence map in main text (key quantitative claims and their exact evidence files/keys).}
  \label{tab:claim_evidence_compact}
  \vspace{2pt}
  \begin{tabular}{p{0.36\linewidth}p{0.13\linewidth}p{0.44\linewidth}}
\toprule
Claim (main text) & Value & Evidence source \\
\midrule
gpt2-medium clean score (CacheMedic++) & 0.2466 & \path{evidence/canonical/medium_contr/metrics/task_metrics.json} \newline \texttt{metrics.clean\_regression.cachemedic} \\
gpt2-medium robustness AUC (CacheMedic++) & 0.0401 & \path{evidence/canonical/medium_contr/metrics/task_metrics.json} \newline \texttt{metrics.robustness\_auc.cachemedic} \\
Paired ablation AUC (no\_contr $\rightarrow$ contr) & 0.0388 $\rightarrow$ 0.0401 & \path{evidence/canonical/medium_no_contr/metrics/task_metrics.json} and \path{evidence/canonical/medium_contr/metrics/task_metrics.json} \newline \texttt{metrics.robustness\_auc.cachemedic} \\
Paired sensitivity ratios ($\delta=1,2$) & 0.9291, 0.9266 & \path{evidence/canonical/medium_no_contr/metrics/stability_metrics.json} and \path{evidence/canonical/medium_contr/metrics/stability_metrics.json} \newline \texttt{metrics["stability.logit\_sensitivity"]} \\
Cross-holdout paired effect (orth\_rotation, 3 seeds) & AUC $\Delta=-0.0003\pm0.0003$; ratios 0.9790, 0.9688 & computed from \path{evidence/seedrep_orth/*/metrics/task_metrics.json} and \path{evidence/seedrep_orth/*/metrics/stability_metrics.json} \\
gpt2-large clean score / robustness AUC (CacheMedic++) & 0.3001 / 0.0475 & \path{evidence/second_model/gpt2_large_contr/metrics/task_metrics.json} \newline \texttt{metrics.clean\_regression.cachemedic}, \texttt{metrics.robustness\_auc.cachemedic} \\
\bottomrule
\end{tabular}

\end{table}

\subsection{Paired contraction ablation}
Table~\ref{tab:ablations} isolates the contraction regularizer by toggling only $\lambda_{\mathrm{contr}}$.
On the canonical seed, contraction improves robustness AUC and reduces logit sensitivity relative to the no-contraction variant at both $\delta=1$ and $\delta=2$.
We treat this as a single-seed directional signal and test its reliability via multi-seed replication below.

\begin{table}[t]
  \centering
  \caption{Paired contraction ablation on gpt2-medium (same operator and training setup; only $\lambda_{\mathrm{contr}}$ differs). ``Stability ratio'' is $\mathrm{Sens}_{\mathrm{contr}}(\delta)/\mathrm{Sens}_{\mathrm{no\ contr}}(\delta)$; values below 1 indicate improved stability.}
  \label{tab:ablations}
  \vspace{2pt}
  \begin{tabular}{lr}
\hline
Setting & Value \\
\hline
repair.family & A \\
repair.rank & 8 \\
repair.apply\_to & KV \\
train.lambda\_contr & 0.0000 \\
train.lambda\_id & 1.5000 \\
\hline
\end{tabular}

\end{table}

\subsection{Five-seed replication (contr vs no\_contr)}
To test run-to-run stability, we extended the paired configuration to five seeds (1234, 2026, 2027, 2028, 2029).
Table~\ref{tab:seedrep} reports mean $\pm$ std over the 5-seed paired set.
The average robustness AUC gain is near zero ($+0.0000 \pm 0.0009$), and mean sensitivity ratios are close to neutral ($1.0086 \pm 0.1345$ at $\delta=1$, $1.0240 \pm 0.1681$ at $\delta=2$), indicating that contraction effects are not consistently directional across seeds in this setup.
In this regime, distillation-only repair already captures most of the task-robustness benefit; explicit contraction does not show a stable average-case gain.

\begin{table}[t]
  \centering
  \small
  \caption{Five-seed replication for the paired contraction ablation (same tuned config, seeds 1234/2026/2027/2028/2029). Reported as mean $\pm$ std.}
  \label{tab:seedrep}
  \vspace{2pt}
  \begin{tabular}{lrrr}
\toprule
Metric & no\_contr ($\mu \pm \sigma$) & contr ($\mu \pm \sigma$) & Effect ($\mu \pm \sigma$) \\
\midrule
Clean score & $0.2440 \pm 0.0048$ & $0.2444 \pm 0.0020$ & $+0.0004 \pm 0.0067$ \\
Robustness AUC & $0.0394 \pm 0.0007$ & $0.0395 \pm 0.0005$ & $+0.0001 \pm 0.0012$ \\
Logit sensitivity ($\delta=1.0$) & $35.3836 \pm 14.7711$ & $36.3964 \pm 11.3408$ & $+1.0128 \pm 5.2282$ \\
Logit sensitivity ($\delta=2.0$) & $35.4753 \pm 15.4637$ & $37.1999 \pm 11.2837$ & $+1.7246 \pm 6.1917$ \\
\midrule
Sensitivity ratio contr/no\_contr ($\delta=1.0$) & \multicolumn{3}{c}{$1.0584 \pm 0.1628$} \\
Sensitivity ratio contr/no\_contr ($\delta=2.0$) & \multicolumn{3}{c}{$1.0892 \pm 0.2004$} \\
\bottomrule
\end{tabular}

\end{table}

Table~\ref{tab:seedrep_ci} adds bootstrap uncertainty over paired seeds ($B=20{,}000$).
All intervals include no-effect thresholds (AUC gain $=0$, ratio $=1$), reinforcing that the current effect estimate remains statistically inconclusive.

\begin{table}[t]
  \centering
  \caption{Bootstrap uncertainty for the 5-seed paired replication (percentile CI over seed pairs, $B=20{,}000$).}
  \label{tab:seedrep_ci}
  \vspace{2pt}
  \begin{tabular}{lrr}
\toprule
Quantity (paired over seeds) & Mean & 95\% bootstrap CI \\
\midrule
Robustness AUC gain (contr - no\_contr) & $+0.0000$ & $[-0.0006,\ 0.0007]$ \\
Sensitivity ratio contr/no\_contr ($\delta=1.0$) & $1.009$ & $[0.928,\ 1.129]$ \\
Sensitivity ratio contr/no\_contr ($\delta=2.0$) & $1.024$ & $[0.922,\ 1.175]$ \\
\bottomrule
\end{tabular}

\end{table}

\subsection{Cross-holdout replication: held-out orthogonal\_rotation (3 seeds)}
To test whether the same conclusion holds on a different OOD split, we ran an additional paired replication with held-out \texttt{orthogonal\_rotation} using seeds 1234, 2028, and 2029.
Table~\ref{tab:orth3} summarizes this three-seed paired comparison.

Distillation-only repair (no\_contr) remains positive relative to baselines in this holdout:
mean robustness AUC gain is $+0.0010 \pm 0.0007$ versus no defense and $+0.0006 \pm 0.0006$ versus best heuristic, with clean-score gain $+0.0115 \pm 0.0020$ versus no defense.
By contrast, the incremental contraction effect is again not robust:
mean AUC effect is $-0.0003 \pm 0.0003$ and mean sensitivity ratios are $0.9790 \pm 0.1200$ at $\delta=1$ and $0.9688 \pm 0.1109$ at $\delta=2$.
Bootstrap CIs over paired seeds ($B=20{,}000$) include no-effect thresholds (AUC effect $[-0.000640, 0.000000]$, ratios $[0.841,1.058]$ and $[0.841,1.041]$).

\begin{table}[t]
  \centering
  \small
  \caption{Cross-holdout paired replication on held-out \texttt{orthogonal\_rotation} (3 seeds: 1234/2028/2029). Reported as mean $\pm$ std.}
  \label{tab:orth3}
  \vspace{2pt}
  \begin{tabular}{lrrr}
\toprule
Metric & no\_contr ($\mu \pm \sigma$) & contr ($\mu \pm \sigma$) & Effect ($\mu \pm \sigma$) \\
\midrule
Clean score & $0.2489 \pm 0.0020$ & $0.2475 \pm 0.0060$ & $-0.0013 \pm 0.0040$ \\
Robustness AUC & $0.0398 \pm 0.0004$ & $0.0395 \pm 0.0008$ & $-0.0003 \pm 0.0003$ \\
\midrule
Sensitivity ratio contr/no\_contr ($\delta=1.0$) & \multicolumn{3}{c}{$0.9790 \pm 0.1200$} \\
Sensitivity ratio contr/no\_contr ($\delta=2.0$) & \multicolumn{3}{c}{$0.9688 \pm 0.1109$} \\
\bottomrule
\end{tabular}

\end{table}

\subsection{Mini hyper-sensitivity: contraction-weight sweep}
Table~\ref{tab:lambda_sensitivity} provides a compact one-seed hyper-sensitivity check under the same tuned setup for $\lambda_{\mathrm{contr}}\in\{0,3,5\}$.
The strongest setting is $\lambda_{\mathrm{contr}}=3.0$.
Pushing contraction to $\lambda_{\mathrm{contr}}=5.0$ degrades clean score and robustness AUC and increases sensitivity beyond both $\lambda_{\mathrm{contr}}=3.0$ and $\lambda_{\mathrm{contr}}=0.0$.
This is an explicit negative result and supports a ``sweet spot'' interpretation rather than monotonic gains from larger contraction weight; however this subsection is single-seed and should be interpreted as hypothesis-generating.

\begin{table}[t]
  \centering
  \caption{Mini hyper-sensitivity on the tuned V-only setup (single-seed comparison).}
  \label{tab:lambda_sensitivity}
  \vspace{2pt}
  \begin{tabular}{rrrrr}
\toprule
$\lambda_{\mathrm{contr}}$ & Clean score & Robustness AUC & Sensitivity ($\delta=1$) & Sensitivity ($\delta=2$) \\
\midrule
0.0 & 0.2400 & 0.0388 & 52.4250 & 53.3308 \\
3.0 & \textbf{0.2466} & \textbf{0.0401} & \textbf{48.7083} & \textbf{49.4163} \\
5.0 & 0.2413 & 0.0390 & 54.5851 & 55.3688 \\
\bottomrule
\end{tabular}

\end{table}

\subsection{Second model: gpt2-large}
Table~\ref{tab:second_model} reports the same evaluation protocol on gpt2-large.
CacheMedic++ improves both clean score and robustness AUC in this single-run confirmation.

\begin{table}[t]
  \centering
  \caption{Second-model confirmation on gpt2-large under the same OOD held-out \texttt{contiguous\_overwrite} protocol.}
  \label{tab:second_model}
  \vspace{2pt}
  \begin{tabular}{lrr}
\toprule
Method & Clean Score & Robustness AUC \\
\midrule
No defense & 0.2974 & 0.0472 \\
Best heuristic (smoothing) & 0.2974 & 0.0472 \\
CacheMedic++ & \textbf{0.3001} & \textbf{0.0475} \\
\bottomrule
\end{tabular}

\end{table}

\subsection{Stability behavior and the ``absolute stability'' limitation}
Figure~\ref{fig:sens} reports logit sensitivity curves and includes the no-contraction ablation.
The canonical pair shows reduced sensitivity for contraction (Table~\ref{tab:ablations}), but replicated aggregates do not preserve a consistent directional advantage across held-out \texttt{contiguous\_overwrite} (Tables~\ref{tab:seedrep} and~\ref{tab:seedrep_ci}) or held-out \texttt{orthogonal\_rotation} (Table~\ref{tab:orth3}).
\textbf{Absolute sensitivity relative to the unmodified baseline also remains higher} in this setting.
We therefore frame CacheMedic++ primarily as a distillation-driven repair method; explicit contraction is currently an inconclusive add-on rather than a robust standalone win.

Figure~\ref{fig:amp} further visualizes that amplification differences are concentrated in the protected layers (10--11), consistent with the operator being active only in that layer subset.

